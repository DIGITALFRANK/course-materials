\documentclass[10pt]{article}
%\setlength{\oddsidemargin}{.5in}
%\setlength{\evensidemargin}{.5in}
\setlength{\textwidth}{5.5in}
\setlength{\topmargin}{0.0in}
\addtolength{\topmargin}{-0.5in}
\setlength{\footskip}{.5in}
\setlength{\textheight}{9.4in}
\linespread{1.1}

% Michael Carlisle, Baruch College, CUNY, MTH 3010, Fall 2017
% github.com/mcarlisle/course-materials (posted Jan 2019)

\usepackage[dvips]{graphicx}
\usepackage{epsfig}
\usepackage{amssymb}
\usepackage{amsmath}

\newcommand{\prob}[1]{\vspace{10mm} \noindent \textbf{Problem #1} \,\,}
\newcommand{\header}{
\begin{center}
Calculus II Quiz \#6
\end{center}

\vspace{2mm}

}

\newcommand{\namefield}{
\noindent Printed Name \underline{\hspace{50mm}} \,\,\, Signature \underline{\hspace{50mm}}
}



\newcommand{\inst}[2]{
Show your work and clearly label your answers on this quiz. \emph{No scrap paper, calculators, or notes are allowed} (or needed). This quiz is scored out of #1 points. (There are #2 points possible.) You have 60 minutes to complete the quiz.

To get credit on a problem, you \emph{must} show work. Even if you can do the work in your head, the point of these exercises is to get you to articulate your thought processes.
}


\begin{document}

\pagenumbering{arabic}

%---:----1----:----2----:----3----:----4----:----5----:----6----:----7----:---

\namefield

\header

\inst{45}{57}

\vspace{10mm} 
(3 pts each test) For each series in \#1-3, use at least three different tests to attempt to answer if the series converges absolutely, converges conditionally, or diverges; if the test fails to give information, also say that. (You must attempt each test at least once over the three problems.) 
\begin{itemize}
\item[(a) ] integral test;
\item[(b) ] comparison test;
\item[(c) ] alternating series test; 
\item[(c) ] ratio test; 
\item[(d) ] root test.
\end{itemize}

\prob{1} $\sum\limits_{n = 1}^{\infty} \frac{3}{2^n}$

\pagebreak

\header

\prob{2} $\sum\limits_{n = 1}^{\infty} \frac{\sin(\frac{n\pi}{2})}{n}$

\vspace{70mm}

\prob{3} $\sum\limits_{n = 1}^{\infty} \frac{(-3)^n}{n}$

%\vspace{60mm}

%\prob{4} $\sum\limits_{n = 1}^{\infty} \frac{5n-2}{2^n}$

\pagebreak

\header

(5+5+5 pts each) For each function, give:
\begin{itemize}
\item[(a) ] the fourth degree MacLaurin polynomial;
\item[(b) ] the fourth degree Taylor polynomial around $a = 4$; 
\item[(c) ] the interval of convergence for the Taylor series in (b). 
\end{itemize}

\prob{4} $f(x) = \ln(x + 2)$

\vspace{65mm} 

\prob{5} $g(x) = \sqrt{x + 5}$



\end{document}
