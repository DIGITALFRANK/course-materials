\documentclass[10pt]{article}
%\setlength{\oddsidemargin}{.5in}
%\setlength{\evensidemargin}{.5in}
\setlength{\textwidth}{5.5in}
\setlength{\topmargin}{0.0in}
\addtolength{\topmargin}{-0.5in}
\setlength{\footskip}{.5in}
\setlength{\textheight}{9.4in}
\linespread{1.1}

% Michael Carlisle, Baruch College, CUNY, MTH 3010, Fall 2017
% github.com/mcarlisle/course-materials (posted Jan 2019)

\usepackage[dvips]{graphicx}
\usepackage{epsfig}
\usepackage{amssymb}
\usepackage{amsmath}

\newcommand{\prob}[1]{\vspace{10mm} \noindent \textbf{Problem #1} \,\,}
\newcommand{\header}{
\begin{center}
Calculus II Sample Final Exam
\end{center}

\vspace{2mm}

}

\newcommand{\namefield}{
Printed Name \underline{\hspace{50mm}} \,\,\, Signature \underline{\hspace{50mm}}
}


\newcommand{\intro}{
\noindent There are a total number of 110 points available (100 are needed for a grade of 100\%).

\noindent Show your work and clearly label your answers. \emph{No scrap paper, calculators, or notes are allowed}. 

\noindent To get credit on a problem, you \emph{must} give a \emph{clear, well-written} explanation. An answer alone will not suffice.

\noindent You have 2 hours to complete this test. 
%\vspace{5mm}
\bigskip

\noindent READ, AND PRINT AND SIGN YOUR NAME BEFORE BEGINNING THE TEST. \\
I will neither give nor receive unauthorized assistance on this test. \\
Printed Name \underline{\hspace{50mm}} \,\,\, Signature \underline{\hspace{50mm}}
}



\begin{document}

\pagenumbering{arabic}

%---:----1----:----2----:----3----:----4----:----5----:----6----:----7----:---

\header

%\inst{45}{57}

%\intro


\prob{1} %(10 pts) % 5.3
Compute the equation of the line tangent to the \emph{inverse} of the function $f(x) = 4^{3x-1}$ at the point $(a, f^{-1}(a)) = (16,1)$.

\prob{2} %(10+5 pts) % 5.6
Compute \[ \frac{d}{dx} \left( \cos( \arcsin(x^2) ) \right) \] and evaluate the derivative at $x = 1$.

\prob{3} %(5+10+10 pts) % 7.2, 7.3
\begin{itemize}
\item[(a) ] Sketch the graph of the region contained by the curves 
\[ y = x^3, \,\, x = 0, \,\, y=0, \,\, y = 8. \] 
\item[(b) ] Compute the volume of the solid obtained by revolving the curve in (a) around the $x$-axis.
\end{itemize}

\prob{4} %(10 pts) % 8.3
\[ \int_0^{\pi} \sin^3(3x) \cos^2(3x) dx =  \]

\prob{5} %(10 pts) % 8.4
\[ \int \sqrt{100 - x^2} dx =  \]

\prob{6} %(5+5 pts) % 8.5
\begin{itemize}
\item[(a) ] 
\[ \lim_{x \to 0} \frac{\sqrt{25-x^2} - 5}{x^2} = \]
\item[(b) ] 
\[ \lim_{x \to \frac{\pi}{2}} \frac{\cos(5x)}{\cos(3x)} = \]
\end{itemize}

\prob{7} %(10 pts) % 8.8
\[ \int_1^{\infty} \frac{\ln(x)}{x^2} dx = \]

\pagebreak

\header

% 9.1
%(5+3+2 pts each) 
For each sequence in Problems 8 and 9, 
\begin{itemize}
\item[(a) ] say if the sequence converges (and to what limit) or diverges (and how);
\item[(b) ] say if the sequence is: bounded/unbounded, strictly/monotonic increasing/decreasing.
\end{itemize}

\prob{8} $b_n = \frac{7 n^2}{n^2 + 1}$, $n \geq 1$

\prob{9} $c_1 = 0.1$, $c_2 = 3$, $c_{n+2} = c_n c_{n+1}$, $n \geq 1$

%\prob{12} \[ \sum_{n = 0}^{\infty} \left( -\frac{2}{5} \right)^n \]

\vspace{5mm}

\prob{10} \[ \sum_{n = 1}^{\infty} \ln\left(\frac{n+2}{n+1}\right) = \]

\prob{11} 
\begin{itemize}
\item[(a) ] Rewrite $6.\overline{24}$ as a geometric series.
\item[(b) ] Rewrite $6.\overline{24}$ as a ratio of whole numbers (in lowest terms).
\end{itemize}


%(4+4+2 pts each) 

%\prob{15} \[ \sum_{n=2}^{\infty} \frac{6n^2 - 5}{6n^3 - 15n + 2}  \]

\prob{12} Test the series for convergence. If it converges, compute the limit or give bounds within 0.25 of the correct value.
\[ \sum_{n = 1}^{\infty} \frac{(-1)^n}{n^2} = \]

%\vspace{60mm}

%\prob{4}  \[ \sum_{n = 1}^{\infty} \frac{5n-2}{2^n} \]

%(5+5+5 pts each) 
\prob{13} For \[ f(x) = 2 e^{3x-1}, \]
give:
\begin{itemize}
\item[(a) ] the fourth degree MacLaurin polynomial;
\item[(b) ] the fourth degree Taylor polynomial around $a = 2$; 
\item[(c) ] the interval of convergence for the Taylor series around $a=2$. 
\end{itemize}


%\prob{13} \[ f(x) = 2\ln(x + 2) \]



\end{document}
