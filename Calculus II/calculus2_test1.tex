\documentclass[10pt]{article}
%\setlength{\oddsidemargin}{.5in}
%\setlength{\evensidemargin}{.5in}
\setlength{\textwidth}{5.5in}
\setlength{\topmargin}{0.0in}
\addtolength{\topmargin}{-0.5in}
\setlength{\footskip}{.5in}
\setlength{\textheight}{9.4in}
\linespread{1.1}

% Michael Carlisle, Baruch College, CUNY, MTH 3010, Fall 2017
% github.com/mcarlisle/course-materials (posted Jan 2019)

\usepackage[dvips]{graphicx}
\usepackage{epsfig}
\usepackage{amssymb}
\usepackage{amsmath}

\newcommand{\prob}[1]{\vspace{10mm} \noindent \textbf{Problem #1} \,\,}
\newcommand{\header}{
\begin{center}
Calculus II Test \#1
\end{center}

\vspace{2mm}

}

\newcommand{\intro}{
\noindent There are a total number of 120 points available (100 are needed for a grade of 100\%).

\noindent Show your work and clearly label your answers. \emph{No scrap paper, calculators, or notes are allowed}. 

\noindent To get credit on a problem, you \emph{must} give a \emph{clear, well-written} explanation. An answer alone will not suffice.

\noindent You have 90 minutes to complete this test. 
%\vspace{5mm}
\bigskip

\noindent READ, AND PRINT AND SIGN YOUR NAME BEFORE BEGINNING THE TEST. \\
I will neither give nor receive unauthorized assistance on this test. \\
Printed Name \underline{\hspace{50mm}} \,\,\, Signature \underline{\hspace{50mm}}
}

 

\begin{document}

\pagenumbering{arabic}

%---:----1----:----2----:----3----:----4----:----5----:----6----:----7----:---

\header

\intro

\vspace{20mm}

\prob{1} ((2+3+5)x3 pts) Give the domain, range, and inverse of each of the following functions: 
\begin{itemize}
\item[(a) ] $f(x) = \ln(x^3 + 2)$
\item[(b) ] $g(x) = \sqrt{9 - x^2}$
\item[(c) ] $y = \arctan\left(\frac{x}{2}\right)$
\end{itemize}

\pagebreak 

\header

\prob{2} (5+5 pts)
\begin{itemize}
\item[(a) ] State and prove the Inverse Derivative Theorem.
\item[(b) ] Use (a) to find the slope of the line tangent to the inverse of $y = x^2 - 5$ at the point $(a, f^{-1}(a)) = (-4,1)$. 
\end{itemize}

\pagebreak

\header

\prob{3} (10 pts) % 5.3

Compute the equation of the line tangent to the \emph{inverse} of the function 
\[ f(x) = 3 \arcsin(x-1) \] at the point $(a, f^{-1}(a)) = (\frac{\pi}{2}, \frac{3}{2})$. 

\pagebreak

\header

\prob{4} (10+5 pts) % 5.6
\begin{itemize}
\item[(a) ] Compute \[ \frac{d}{dx} \left( \sin( \arctan(x^2) ) \right) \]
and write your answer with \emph{no trigonometric functions} in it.
\item[(b) ] Evaluate the derivative found in (a) at $x = \frac{1}{2}$.
\end{itemize}

\pagebreak

\header

\prob{5} (10 pts) % 5.7
Compute \[ \int_0^1 \frac{x^2}{x^6 + 6} dx. \]

\pagebreak 

\header

\prob{6} (5+5 pts) % 7.1

\begin{itemize}
\item[(a) ] Sketch the graph of the region completely enclosed by the curves 
\begin{align*}
f(x) & = 13x+12 \text{ and } g(x) = x^3.
\end{align*}
(Hint: Guess one of the intersection points, and use that to help find the others, \emph{before} sketching. How many intersections are there?)
\item[(b) ] Compute the area of the region sketched in (a).
\end{itemize}


\pagebreak

\header

\prob{7} (5+10+10+10 pts) % 7.2, 7.3, 7.4

\begin{itemize}
\item[(a) ] Sketch the graph of the region contained by the curves 
\[ y = 0, \,\, y = (x-4)^2 + 1, \,\, x = 2, \,\, x = 6. \] 
\item[(b) ] Compute the volume of the solid obtained by revolving the region in (a) around the $x$-axis \emph{using the washer method}.
\item[(c) ] Compute the volume of this solid from (a) \emph{using the shell method}. \\
(Hint: compute the volume of \emph{half} of the solid, and mulitply your answer by 2.)
\item[(d) ] Set up, but do not compute, the surface area of this solid.
\end{itemize}



\end{document}
