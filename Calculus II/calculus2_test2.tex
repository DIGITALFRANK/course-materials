\documentclass[10pt]{article}
%\setlength{\oddsidemargin}{.5in}
%\setlength{\evensidemargin}{.5in}
\setlength{\textwidth}{5.5in}
\setlength{\topmargin}{0.0in}
\addtolength{\topmargin}{-0.5in}
\setlength{\footskip}{.5in}
\setlength{\textheight}{9.4in}
\linespread{1.1}

% Michael Carlisle, Baruch College, CUNY, MTH 3010, Fall 2017
% github.com/mcarlisle/course-materials (posted Jan 2019)

\usepackage[dvips]{graphicx}
\usepackage{epsfig}
\usepackage{amssymb}
\usepackage{amsmath}

\newcommand{\prob}[1]{\vspace{10mm} \noindent \textbf{Problem #1} \,\,}
\newcommand{\header}{
\begin{center}
Calculus II Test \#2
\end{center}

\vspace{2mm}

}

\newcommand{\intro}{
\noindent There are a total number of 110 points available (100 are needed for a grade of 100\%).

\noindent Show your work and clearly label your answers. \emph{No scrap paper, calculators, or notes are allowed}. 

\noindent To get credit on a problem, you \emph{must} give a \emph{clear, well-written} explanation. An answer alone will not suffice.

\noindent You have 90 minutes to complete this test. 
%\vspace{5mm}
\bigskip

\noindent READ, AND PRINT AND SIGN YOUR NAME BEFORE BEGINNING THE TEST. \\
I will neither give nor receive unauthorized assistance on this test. \\
Printed Name \underline{\hspace{50mm}} \,\,\, Signature \underline{\hspace{50mm}}
}

 

\begin{document}

\pagenumbering{arabic}

%---:----1----:----2----:----3----:----4----:----5----:----6----:----7----:---

\header

\intro

\vspace{20mm}

\prob{1} (10+5 pts)

\begin{itemize}
\item[(a) ] $\int x^2 \cos(5x) dx = $ \\
\item[(b) ]$\int_0^{\pi} x^2 \cos(5x) dx = $
\end{itemize}

\pagebreak 

\header

\prob{2} (10 pts)

\[ \int \frac{2}{\sqrt{81-4x^2}} dx =  \]

\pagebreak

\header

\prob{3} (10 pts) 

\[ \int \sin(2x)^6 \cos(2x)^3 dx =  \]

\pagebreak

\header

\prob{4} (10 pts) 

\[ \int \frac{x}{84x^2 - 1} dx = \]

\pagebreak

\header

\prob{5} (10+5 pts) 

Use Simpson's Rule, with $n=6$, to approximate $\int_0^2 (x^2 + 4) dx$, and compare to the exact answer of the integral.

\[ \text{Hint: } \,\,\,\,\, \int_a^b f(x) dx \approx \frac{b-a}{3n}[f(x_0) + 4 f(x_1) + 2 f(x_2) + 4 f(x_3) + \cdots + 4 f(x_{n-1}) + f(x_n)]. \]

\pagebreak 

\header

\prob{6} (10+10 pts) 

\begin{itemize}
\item[(a) ] $\lim\limits_{x \to 3} \frac{e^{3-x}}{(x+3)^2} = $
\item[(b) ] $\lim\limits_{x \to \pi} \frac{\sin(6x)}{\sin(3x)} = $
\end{itemize}

\pagebreak

\header

\prob{7} (10+10 pts) 

\begin{itemize}
\item[(a) ] Find the area of the region bounded by $y = 5$ and $y = 5 - x^2 e^{-x}$.
\item[(b) ] Compute the volume of the solid formed by revolving the curve $y = e^{-x}$, starting at $x=1$ and going right (limit as $x \to \infty$) around the $x$-axis.
\end{itemize}

\pagebreak

\header

\prob{8} (10 pts)

Compute the area of the region bounded by the three curves 
\[ y = \sqrt{4 - x^2}, \,\, y = \sqrt{1 - (x-1)^2}, \,\, \text{ and } y = -\sqrt{1 - (x+1)^2}. \]
(Hint: Think geometrically.)


\end{document}
