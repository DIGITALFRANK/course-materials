\documentclass[10pt]{article}
\setlength{\oddsidemargin}{.5in}
\setlength{\evensidemargin}{.5in}
\setlength{\textwidth}{6.0in}
\setlength{\topmargin}{0.0in}
\addtolength{\topmargin}{-1.0in}
\setlength{\footskip}{.5in}
\setlength{\textheight}{9.5in}
\linespread{1.1}

\usepackage[dvips]{graphicx}
\usepackage{epsfig}
\usepackage{amssymb}
\usepackage{amsmath}
\usepackage{mjcquiz}

% Michael Carlisle, Baruch College, CUNY, MTH 4100, Spring 2018
% github.com/mcarlisle/course-materials (posted Jan 2019)

\newcommand{\examprobs}{
\textbf{NOTE}: Graphing calculators are \emph{not} allowed. 

\vspace{10mm}

For each of the following systems of equations, do the following: 
\begin{itemize}
\item[(i) ] (5 pts) Convert the system into a matrix equation of the form $A \vec{x} = b$. 
\item[(ii) ] (5 pts) Give the dot product of each pair of column vectors of $A$. \\
(For example, if $A$ has four columns, then you will compute ${4 \choose 2} = 6$ dot products.)
\item[(iii) ] (5 pts) Give the unit vector in the direction of each column vector of $A$. 
\item[(iv) ] (10 pts) Attempt to solve the system $A \vec{x} = b$. Clearly state your result, whether the system has no solution, a unique solution, or infinitely many solutions.
\item[(v) ] (10 pts) If the system has a unique solution, compute the factorization $A = LDU$ and the inverse $A^{-1}$. If the system does not have a unique solution, state why this factorization does not exist.
\end{itemize}
}

\newcommand{\answershere}[1]{\vspace{130mm}
\begin{center}
\line(1,0){400}
\end{center}
\textbf{Write answers for Problem #1 clearly here:}}


\begin{document}

\pagenumbering{arabic}



%---:----1----:----2----:----3----:----4----:----5----:----6----:----7----:---

\namefield

\headerexam{Linear Algebra}{Exam \#1}

\instexam{100}{120}

\examprobs

\pagebreak

\headerexam{Linear Algebra}{Exam \#1}

\prob{1} 

\begin{align*}  % unique solution
-x + 3y + -2z & = 7 \\
-4x + 14y - 6z & = -10 \\
3x - 10y  & = 4
\end{align*}

%\answershere{1}

\pagebreak


\headerexam{Linear Algebra}{Exam \#1}

\prob{2}

\begin{align*}  % no solutions
-x + 3y + -2z & = 7 \\
-4x + 14y - 6z & = -10 \\
3x - 10y + 5z & = 4
\end{align*}

%\answershere{2}

\pagebreak


\headerexam{Linear Algebra}{Exam \#1}

\prob{3}

\begin{align*}  % infinitely solutions
-x + 3y + -2z & = 0 \\
-4x + 14y - 6z & = 0 \\
3x - 10y + 5z & = 0
\end{align*}

%\answershere{3}


\end{document}
