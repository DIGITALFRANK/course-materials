\documentclass{beamer}

\usepackage{mjclectureslides}


\title[Intro to Vectors, Solving Linear Equations]{Linear Algebra and Matrix Methods \\ Chapters 1-2: \\
Introduction to Vectors, \\ Solving Linear Equations }
%\author[Prof. Michael Carlisle]{Prof. Michael Carlisle}
%\institute{Baruch College, CUNY}
%\date{Spring 2018}
\date{}

%---:----1----:----2----:----3----:----4----:----5----:----6----:----7----:---

\begin{document}

\frame{\titlepage}



\frame{ \frametitle{A simple example: two equations in two variables}

How do we solve this system of equations? 

\begin{align*}
4x + 2y & = 6 \\
-10x + 5y & = -20
\end{align*}

\vspace{5mm}

Right at the outset, the question is tricky... 

\begin{center}
IS there a solution to this system of equations? 
\end{center}

} 


\frame{ \frametitle{A simple example: two equations in two variables}

How do we solve this system of equations? 

\begin{align*}
4x + 2y & = 6 \\
-10x + 5y & = -20
\end{align*}

\vspace{5mm}

There are three possible cases: 
\begin{itemize}
\item There are NO SOLUTIONS.
\item There is EXACTLY ONE SOLUTION.
\item There are INFINITELY MANY SOLUTIONS.
\end{itemize}



}


\frame{ \frametitle{Vectors in two dimensions}

We are going to analyze a lot of this kind of system. 

\vspace{5mm}

An ordered pair $(x,y)$ is a \textbf{vector} in the two-dimensional plane of real numbers $\R^2$. 

\vspace{5mm}

When written like 
\[ \rvv{x}{y} \]
\hspace{55mm} we call this a \textbf{row vector}. 

\vspace{5mm}

When written like 
\[ \cvv{x}{y} \]
\hspace{55mm} we call this a \textbf{column vector}. 

}


\frame{ \frametitle{Considering each equation}

Look at the first equation: 
\begin{align*}
4x + 2y & = 6 
\end{align*}

\vspace{5mm}

There are an infinite number of solutions $(x,y)$ to this single equation: these solutions form a \emph{line} in the plane.
\begin{align*}
L_1 = \{ (x, y): \,\, 4x + 2y = 6  \}
\end{align*}

}


\frame{ \frametitle{Considering each equation}

Likewise, there are an infinite number of solutions $(x,y)$ to the second equation, all of which also form a line in the plane.
\begin{align*}
L_2 = \{ (x, y): \,\, -10x + 5y = -20  \}
\end{align*}

\vspace{5mm}

The solution of the system of equations is their intersection: 
\[ L_1 \cap L_2. \]

}


\frame{ \frametitle{Considering each variable}

We can consider all coefficients for one variable, and the right hand side, as column vectors. 

\vspace{5mm}

This transforms the system of equations into one equation, where the variables are multiplied by vectors: 

\vspace{5mm}

When written like this, 
\[ x \cvv{4}{-10} + y \cvv{2}{5} = \cvv{6}{-20}, \]
our system is now one equation, with vector arithmetic.

}


\frame{ \frametitle{What is a vector?}

But what is this vector arithmetic we see here?

\vspace{10mm}

 ... wait, what even \emph{is} a vector?

}


\frame{ \frametitle{Vector arithmetic: scalar multiplication}

A vector $(x,y)$ in two dimensions is an arrow, starting from the origin, and pointing at the point $(x,y)$ in the plane.

\vspace{10mm}

\textbf{Scalar multiplication} of a vector by a real number is \emph{scaling} the vector: stretching or shrinking it to a different length. The real number doing the scaling is called (you guessed it) a \textbf{scalar}. 

}


\frame{ \frametitle{Vector arithmetic: scalar multiplication}

To do scalar multiplication, multiply each coordinate by the scalar. 

\[ 5.2 \cvv{4}{-10} = \cvv{5.2 \cdot 4}{5.2 \cdot -10} = \cvv{20.8}{-52} \]

\vspace{10mm} 

If the scalar is negative, the vector ``changes'' ``direction''. 

\[ -3.1 \cvv{5}{2} = \cvv{-3.1 \cdot 5}{-3.1 \cdot 2} = \cvv{-15.5}{-6.2} \]

}


\frame{ \frametitle{Vector arithmetic: vector addition}

\textbf{Vector addition} is done by adding the coordinates of each vector. 

\[ \cvv{4}{-10} + \cvv{9}{3} = \cvv{4 + 9}{-10 + 3} = \cvv{13}{-7} \]

\vspace{5mm}

Graphically, we can draw the second vector as starting at the tip of the first vector; the sum is where the second vector now ends. 

}


\frame{ \frametitle{Vector arithmetic: vector addition}

The sum of two vectors is called a \textbf{linear combination}. 

\vspace{5mm}

Vector addition, like ``regular'' scalar addition, is \textbf{commutative}: 

\[ \cvv{4}{-10} + \cvv{9}{3} = \cvv{4 + 9}{-10 + 3} = \cvv{9 + 4}{3 + -10} = \cvv{9}{3} + \cvv{4}{-10}. \]

}


\frame{ \frametitle{Solving a system (column view): scalars = variables}

To solve the column vector equation 
\[ x a_1 + y a_2 = b, \]
where 
\[ a_1 = \cvv{4}{-10}, a_2 = \cvv{2}{5}, b = \cvv{6}{-20}, \]
we ask, 

\begin{center}
How do we scale $a_1$ and $a_2$, then add them, to get $b$?
\end{center}

A more ``linear algebra''-type way of saying this is, 

\begin{center}
What \textbf{linear combination} of $a_1$ and $a_2$ is $b$?
\end{center}

}


\frame{ \frametitle{Dot product of two vectors}

The \textbf{dot product} of two vectors of the same size is the sum of the products of their coordinates, in sequence. 

\vspace{5mm}

In two dimensions, consider two vectors $(1, 6)$ and $(-3, 5)$. 

\vspace{5mm}

Their dot product is 

\[ (1, 6) \cdot (-3, 5) = (1 \cdot -3) + (6 \cdot 5) = -3 + 30 = 27. \]

}


\frame{ \frametitle{Length (norm) of a vector}

The \textbf{length} (\textbf{Euclidean norm}) of a vector is the square root of its dot product. We use doubled absolute value bars to denote length.

\[ ||(1,6)|| = \sqrt{(1,6) \cdot (1,6)} = \sqrt{1^2 + 6^2} = \sqrt{37}. \] 

}


\frame{ \frametitle{Dot product, angle between vectors}

In general, if $v = \rvvv{v_1}{v_2 ...}{v_n}$ and $w = \rvvv{w_1}{w_2 ...}{w_n}$ are two vectors in $\R^n$, their dot product can be written in the summation notation 

\[ v \cdot w = \sum_{i=1}^n v_i w_i. \]

\vspace{5mm}

This number is related to the cosine of the angle $\theta$ between the two vectors $v$ and $w$, regardless of the dimension of the space of their vectors: 

\[ \cos(\theta) = \frac{v \cdot w}{||v|| \cdot ||w||}. \]

}


\frame{ \frametitle{Cauchy-Bunyakovsky-Schwarz inequality}

In fact, since $-1 \leq \cos(\theta) \leq 1$ for any angle $0 \leq \theta < 2\pi$, there is a result called the \textbf{Cauchy-Bunyakovsky-Schwarz inequality}: 

\[ |v \cdot w| \leq ||v|| \cdot ||w||. \]

}


\frame{ \frametitle{Orthogonal vectors}

A special case of the dot product is when $v \cdot w = 0$; in this case $\cos(\theta) = 0$, and so $v$ and $w$ make a right angle. In this case, we call $v$ and $w$ \textbf{orthogonal} or \textbf{perpendicular}, and use the notation 

\[ v \perp w \iff v \cdot w = 0. \]

\vspace{5mm}

Note that the \textbf{zero vector}, the vector of all 0 (for any size), is orthogonal to any other vector of its size.

}


\frame{ \frametitle{Unit, orthonormal, standard vectors}

If the length of a vector $v$ is $||v|| = 1$, we call it a \textbf{unit} vector.

\vspace{5mm}

Two orthogonal unit vectors $v$ and $w$ are called \textbf{orthonormal}. 

\vspace{5mm}

The \textbf{standard unit vectors} in $\R^n$ are $n$ vectors, each with a 1 in one coordinate, and 0 in every other coordinate. We denote them 

\[ \{e_1, e_2, ..., e_n\}. \]

}


\frame{ \frametitle{Unit, orthonormal, standard vectors}

For the plane $\R^2$, these are 
\[ e_1 = \cvv{1}{0}, e_2 = \cvv{0}{1} \]
and in three-dimensional space $\R^3$, these are 
\[ e_1 = \cvvv{1}{0}{0}, e_2 = \cvvv{0}{1}{0}, e_3 = \cvvv{0}{0}{1}. \]


}



\frame{ \frametitle{Length = Distance from 0 to end = Pythagorean Theorem}

The length of a vector is the distance from the point 0 = (0,0) to the point at the end of the vector. 

\vspace{10mm}

This is the hypotenuse of a right triangle that should be familiar from trigonometry: use the Pythagorean Theorem (also known as the Distance Formula) to find the length.

}


\frame{ \frametitle{Length = Distance from 0 to end = Pythagorean Theorem}

If two vectors $v$ and $w$ are orthogonal, then the Pythagorean Theorem can also be used to describe the length of their difference: 

\vspace{5mm}

\[ v \perp w \implies ||v||^2 + ||w||^2 = ||v - w||^2 \]

}



\frame{ \frametitle{Vector: direction and length, Triangle Inequality}

For any nonzero vector $v$, 
\[ u = \frac{1}{||v||} v \]
is a unit vector.

\vspace{5mm}

Thus, a vector encodes the notions of ``length'' and ``direction''. 

\vspace{5mm}

In fact, the difference in directions offers another inequality, since the difference of two vectors creates a triangle: 

\[ \textbf{Triangle Inequality:} \,\, ||v|| + ||w|| \geq ||v-w|| \]

}


\frame{ \frametitle{Matrix view of the system of equations}

A \textbf{matrix} is a row vector of column vectors, lined up together. \\
(Equivalently, a matrix is a column vector of row vectors.)

\vspace{5mm}

We can write the columns of coefficient vectors from the system of equations as 
\[ A = \rvv{a_1}{a_2} = \mrr{4}{2}{-10}{5}. \]

}


\frame{ \frametitle{Matrix view of the system of equations}

We can write the coefficients from each equation similarly as row vectors: 
\[ r_1 = \rvv{4}{2}, \,\, r_2 = \rvv{-10}{5} \]
giving the same matrix 
\[ A = \cvv{r_1}{r_2} = \mrr{4}{2}{-10}{5}. \]

}


\frame{ \frametitle{Matrix view of the system of equations}

Thus, the system of equations 
\begin{align*}
4x + 2y & = 6 \\
-10x + 5y & = -20
\end{align*}
can be written, placing the variables in a column vector 
\[ \vec{x} = \cvv{x}{y}, \] 
as one matrix-vector equation 
\[ A\vec{x} = b. \]
To understand this, we'll need to explain \textbf{matrix multiplication}. 

}


\frame{ \frametitle{Matrix multiplication}

The \textbf{dimensions} of a matrix are 
\begin{center}
\# rows $\times$ \# columns, 
\end{center}
which we will typically write as $m \times n$. We can multiply two matrices $A$ and $B$, and get product $AB$ (the order is important here), only if the \# of rows of $A$ = \# of columns of $B$. 

\vspace{5mm}

That is, if $A$ is $m \times n$ and $B$ is $r \times s$, then the matrix product $AB$ only exists if $n = r$. The dimensions of $AB$ are $m \times s$.

\vspace{5mm}

Thus, matrix multiplication is NOT commutative; in general, the products $AB$ and $BA$ are different matrices, and only both exist if $m = s$ and $n = r$.

}


\frame{ \frametitle{Matrix multiplication}

But what \emph{is} the matrix product $AB$? 

\vspace{5mm}

Let us assume $A$ is $m \times n$ and $B$ is $n \times p$, so that the matrix product $M = AB$ exists. $M$ is $m \times p$. 

\vspace{5mm}

Writing the rows of $A$ with notation $r_i$, $i = 1, 2, ..., m$ and the columns of $B$ as $c_j$, $j=1, 2, ..., p$, 

\[ A = \cvvvv{r_1}{r_2}{...}{r_m}, \,\, B = \rvvvv{c_1}{c_2}{...}{c_p} \]

}


\frame{ \frametitle{Matrix multiplication}

... then the entry of $M$ in row $i$, column $j$ is the dot product 
\[ m_{ij} = r_i \cdot c_j = \sum_{k=1}^n a_{ik} b_{kj}, \]
where $a_{ik}$ is entry $k$ in row $i$, and $b_{kj}$ is entry $k$ in column $j$. 

\vspace{5mm}

We denote $M = (m_{ij})_{m \times p}$ to emphasize the (row, col) structure. 

}


\frame{ \frametitle{Matrix multiplication}

\begin{center}
A row vector in $\R^n$ is a $1 \times n$ matrix; \\
a column vector in $\R^n$ is a $n \times 1$ matrix.
\end{center}

\vspace{5mm}

If $M = AB$ as previously described, we can consider $M$ to be a row vector of matrix-vector products: 
\[ M = \rvvvv{Ac_1}{Ac_2}{...}{Ac_p} = \cvvvv{r_1 B}{r_2 B}{...}{r_m B}. \]

}


\frame{ \frametitle{Scalar multiplication}

Now that we've addressed the hard part of matrix arithmetic, the easy parts are just like vector arithmetic: 

\begin{itemize}
\item \textbf{scalar multiplication}: A matrix can be scaled by a real number by multiplying each element by that number. 

\vspace{5mm}

If $c \in \R$ and $A$ is a matrix with entry $a_{ij}$ in row $i$, column $j$, then $cA$ is the matrix with entry $c a_{ij}$ in that position. 

\vspace{5mm}

For example, if 
\[ A = \begin{pmatrix} 2 & -4 & 5 \\ 6 & 6 & 7 \end{pmatrix}, \]
then 
\[ -3A = \begin{pmatrix} -6 & 12 & -15 \\ -18 & -18 & -21 \end{pmatrix}. \]
\end{itemize} 

}


\frame{ \frametitle{Matrix addition}

\begin{itemize}
\item \textbf{matrix addition}: Two matrices of the same dimensions can be added by adding their corresponding elements.

\vspace{5mm}

For example, if 
\[ A = \begin{pmatrix} 2 & -4 & 5 \\ 6 & 6 & 7 \end{pmatrix} \text{ and }  
B = \begin{pmatrix}10 & 8 & 0 \\ -1 & -9 & 2\end{pmatrix}, \]
then 
\[ A+B = \begin{pmatrix}12 & 4 & 5 \\ 5 & -3 & 9\end{pmatrix}. \]

\vspace{5mm}

Note that two matrices with different dimensions cannot be added together.
\end{itemize} 

}


\frame{ \frametitle{Matrix view of the system of linear equations}

Back to the view of a system of equations 

\vspace{5mm}

\begin{align*}
4x + 2y & = 6 \\
-10x + 5y & = -20
\end{align*}

\vspace{5mm}
as 

\vspace{5mm}

\[ A\vec{x} = b... \]

}


\frame{ \frametitle{Matrix view of the system of linear equations}

In the row view, this product is 

\vspace{5mm}

\[ \cvv{r_1 \cdot \vec{x}}{r_2 \cdot \vec{x}} = \cvv{b_1}{b_2}. \]

\vspace{5mm}

In the column view, this product is 

\vspace{5mm}

\[ x a_1 + y a_2 = b. \]

\vspace{5mm}

These views all mean the same thing.

}


\frame{ \frametitle{Solving a system of linear equations}

For this simple $2 \times 2$ system, we are familiar with solving. 

\vspace{5mm}

What if the system was much larger? $10 \times 10$, or $200 \times 200$?

\vspace{5mm}

We need a general technique that we will discover from the simplest case: a $1 \times 1$ ``system''.

\vspace{5mm}

Let $a, b \in \R$ with $a \neq 0$. What is the solution $x$ of the equation 
\[ ax = b \,\, ? \]
This is easy: 
\[ x = \frac{b}{a}. \]

}


\frame{ \frametitle{Solving a system of linear equations... }

This fraction $\frac{b}{a}$ is the product $a^{-1} b$, where $a^{-1}$ is the reciprocal \textbf{(multiplicative inverse)} of $a \in \R$: the number such that 

\vspace{5mm}

\[ a^{-1} a = a a^{-1} = 1. \]

\vspace{10mm}

1 is the \textbf{multiplicative identity} of the real numbers. 

}


\frame{ \frametitle{Solving a system of linear equations... }

But be careful here: we know, for real numbers, 

\[ a^{-1} b = b a^{-1} \]

\vspace{2mm}

since regular real number multiplication is commutative. 

\vspace{5mm}

Matrix multiplication is \emph{not} commutative. (It \emph{is} associative.)

\vspace{5mm}

We will stick to the order of terms with which we are presented.

}


\frame{ \frametitle{... requires a matrix inverse. }

The solution $x$ to the equation 

\[ ax = b \]

is 

\[ a^{-1} ax = a^{-1} b \implies x = a^{-1} b. \]

\vspace{5mm}

What, then, is the solution to the system $A \vec{x} = b$? 

}


\frame{ \frametitle{... requires a matrix inverse. }

If a \textbf{matrix inverse} $A^{-1}$ exists, a matrix such that 

\vspace{4mm}

\[ A^{-1} A = A A^{-1} = I, \]

\vspace{5mm}

with $I$ the \textbf{identity matrix} (for that size matrix), then the solution to the system is, specifically using \emph{left multiplication},  

\vspace{5mm}

\[ A \vec{x} = b \implies A^{-1} A \vec{x} = A^{-1} b \implies \vec{x} = A^{-1} b. \]

}


\frame{ \frametitle{When does a matrix inverse exist?}

There are many requirements for a matrix inverse $A^{-1}$ to exist.

\vspace{10mm}

First, $A$ must be a \textbf{square} matrix: its dimensions must be $n \times n$.

\vspace{10mm}

Second, the column vectors must be \textbf{linearly independent}; we will define this more rigorously later. 

}


\frame{ \frametitle{When does a matrix inverse exist?}

Let's see a $3 \times 3$ example of each of the three kinds of systems mentioned at the beginning: 

\vspace{5mm}

\begin{itemize}
\item There are NO SOLUTIONS.
\item There is EXACTLY ONE SOLUTION.
\item There are INFINITELY MANY SOLUTIONS.
\end{itemize}


}


\frame{ \frametitle{Dimension of a Space of Real Number-Scaled Vectors}

The \textbf{dimension} of a set of vectors is the maximum number of vectors required to uniquely describe \emph{any} point in the space \emph{as a linear combination}. 
 
\vspace{10mm}

The term ``dimension'' does not apply to the empty set $\emptyset$.

\vspace{10mm}

A set with only one vector has dimension 0: you cannot express any other vectors via linear combination. Geometrically, this is a point.


}


\frame{ \frametitle{Dimension of a Space of Real Number-Scaled Vectors}

If the dimension of the set of vectors is greater than 0, then the set has infinitely many vectors in it. Geometrically, 
\begin{itemize}
\item dimension = 1: line
\item dimension = 2: plane
\item dimension = 3: space
\item dimension $>$ 3: while we still call this ``space'', we can't easily visualize this kind of space.
\end{itemize}

}


\frame{ \frametitle{3 x 3 Systems}

No solutions: 
\begin{align*}
3x + 4y + 5z & = 6 \\
-6x - 8y - 8z & = -8 \\
3x + 4y + 5z & = 9
\end{align*}

Unique solution: 
\begin{align*}
3x + 4y + 5z & = 6 \\
-6x - 8y - 8z & = -8 \\
x - y + 2z & = 5
\end{align*}

Infinitely many solutions: 
\begin{align*}
3x + 4y + 5z & = 6 \\
-6x - 8y - 8z & = -8 \\
z & = 2
\end{align*}


}


\frame{ \frametitle{One equation, three variables: a plane of solutions}

One equation in three variables, such as 
\[ 3x + 4y + 5z = 6, \]
requires two numbers to describe a point in its solution set. 

\vspace{5mm}

That is, the set 
\[ \{ (x,y,z) \in \R^3: 3x + 4y + 5z = 6 \} \]
has dimension 2. 

\vspace{5mm}

Why? If you pick any $x, y \in \R$, then $z$ is determined: 
\[ z = \frac{1}{5} (-3x - 4y + 6). \]

}


\frame{ \frametitle{One equation, three variables: a plane of solutions}

As a system, this one equation in three variables has infinitely many solutions: these solutions constitute a plane. 

\vspace{10mm}

We can rewrite the solution set as 
\[ \left\{ \left(x,y,z\right): x, y \in \R, z = \frac{1}{5} (-3x - 4y + 6) \right\}. \]

}


\frame{ \frametitle{Two equations, three variables: plane, line, or no solution}

By adding a second equation in three variables, making a system of two equations, we may have any of the following scenarios: 

\vspace{5mm}

\begin{itemize}
\item \textbf{redundancy}: If the new equation is a linear combination of previous equations in a system, then the new equation causes no change to the system solution. A shorthand for this scenario is ``0 = 0'' - no new information.
\item \textbf{reduce the solution set dimension by 1}: Two intersecting planes have the common solution set of a line;
\item \textbf{inconsistency}: If the new equation is a parallel plane to a previous equation, then there is no common solution for all equations. A shorthand is ``0 = 1'', a contradiction. 
\end{itemize}

}


\frame{ \frametitle{Two equations, three variables: plane (0 = 0)}

\begin{align*}
3x + 4y + 5z & = 6 \\
9x + 12y + 15z & = 18
\end{align*}
The second equation is the first equation multiplied by 3 on both sides. They are, basically, the same equation. 

\vspace{5mm}

Adding $\frac{1}{3}$ times ``row (equation) 1'' to row 2, the system ``reduces'': 
\begin{align*}
3x + 4y + 5z & = 6 \\
0 & = 0
\end{align*}
and we really only have the one equation system we started with.

}


\frame{ \frametitle{Two equations, three variables: line (dim reduced by 1)}

\begin{align*}
3x + 4y + 5z & = 6 \\
-6x - 8y - 8z & = -8
\end{align*}

These are two intersecting planes. The solution has dimension 1:
\[ \{ (x,y,z): 3x + 4y + 5z = 6 \text{ and } -6x - 8y - 8z = -8 \} \]
For each value of $x$, say $x = c$, the system reduces to a $2 \times 2$ system in $y$ and $z$: 
\begin{align*}
4y + 5z & = 6 + -3c \\
-8y - 8z & = -8 + 6c
\end{align*}
which is a system that has a one-point solution $\{(y,z)\}$, where each is a function of $c$.

}


\frame{ \frametitle{Two equations, three variables: inconsistency (0 = 1)}

\begin{align*}
3x + 4y + 5z & = 6 \\
3x + 4y + 5z & = 9
\end{align*}
These two equations represent parallel planes. 

\vspace{5mm}

If we subtract row 1 from row 2, then the system reduces to 
\begin{align*}
3x + 4y + 5z & = 6 \\
0 & = 3
\end{align*}
which clearly is impossible. 


}



\frame{ \frametitle{Unique solution of a system of linear equations}

What does it take to have a unique, one-point solution to a system of linear equations? 

\vspace{5mm}

Each equation of $n$ variables ideally reduces the solution set dimension by 1. 

\vspace{5mm}

To have a unique solution: 
\begin{itemize}
\item $n$ equations in $n$ variables
\item all $n$ equations are linearly independent \\
(none is a linear combination of any subset of the others)
\end{itemize}

}


\frame{ \frametitle{Unique solution of a system of linear equations}

To have a unique solution: 
\begin{itemize}
\item $n$ equations in $n$ variables
\item all $n$ equations are linearly independent (none is a linear combination of any subset of the others)
\end{itemize}

\vspace{5mm}

To \emph{compute} a unique solution: 
\begin{itemize}
\item reduce the system to a \textbf{triangular system} via the process of \textbf{elimination}, or \textbf{Gauss-Jordan reduction}
\item use \textbf{back substitution} from the bottom to get the solution.
\end{itemize}

}


\frame{ \frametitle{Unique solution $\iff$ inverse matrix exists}

If a system of $n$ equations in $n$ variables can be written as the matrix-vector equation 
\[ A \vec{x} = b, \]
then the system has a unique solution $\vec{x}$ iff $A^{-1}$ exists; then, 
\[ \vec{x} = A^{-1} b. \]
If $A^{-1}$ exists, we call $A$ an \textbf{invertible} or \textbf{nonsingular} matrix. 

\vspace{5mm}

If $A^{-1}$ does not exist, we call $A$ \textbf{non-invertible} or \textbf{singular}. 


}


\frame{ \frametitle{Identity matrix}

If a square matrix $A$ is invertible, then the product 
\[ A A^{-1} = A^{-1} A = I, \]
the \textbf{identity matrix}, which generalizes the concept of the number ``1'' for square matrices. 

\vspace{5mm}

$I$ has 1 on the \textbf{diagonal} (row index = col index) and 0 elsewhere.
\[ I = \mrrr{1}{0}{0}{0}{1}{0}{0}{0}{1} \]

Multiplying a vector or matrix by $I$ leaves it unchanged (much like $1 \cdot 5 = 5$).
\[ I A = \mrrr{1}{0}{0}{0}{1}{0}{0}{0}{1} \mrrr{3}{4}{5}{1}{-1}{2}{-6}{-8}{-8} = \mrrr{3}{4}{5}{1}{-1}{2}{-6}{-8}{-8} \]


}


\frame{ \frametitle{Inverse, Identity matrices}

Note that, in the special case of multplying by an inverse matrix, 
\[ A A^{-1} = A^{-1} A = I, \]
multiplication is \emph{commutative} - you can switch the order.

\vspace{5mm}

Another property of inverse matrices: just like in number division, 
\[ \frac{1}{(1/a)} = a, \]
we have that the inverse of the inverse matrix is the original matrix: 
\[ (A^{-1})^{-1} = A. \]

}


\frame{ \frametitle{Elementary row operations}

Solving a system involves \textbf{elementary row operations}. 

\vspace{5mm}

(Recall, we refer to each equation as a row.) 

\vspace{5mm}

There are three elementary row operations: 
\begin{itemize}
\item swap two rows
\item scale (multiply) a row
\item add a scaled row to another row
\end{itemize}

\vspace{5mm}

For each of these operations, there is an \textbf{elementary matrix}, with one change of $I$, that does this operation via left multiplication on the system.

}


\frame{ \frametitle{Elementary row operations: swap two rows}

To \textbf{swap two rows}, use an elementary \textbf{permutation matrix}. 

\vspace{5mm} 

For example, to \textbf{swap two rows} 2 and 3, use 
\[ P_{2,3} = \mrrr{1}{0}{0}{0}{0}{1}{0}{1}{0} \]
(In practice, we'll avoid this one by handling swaps outside of our solving process. Note that, intuitively, rerranging the equations in a system does not change the solution.) For example, 
\[ P_{2,3} A = \mrrr{1}{0}{0}{0}{0}{1}{0}{1}{0} \mrrr{3}{4}{5}{1}{-1}{2}{-6}{-8}{-8}
 = \mrrr{3}{4}{5}{-6}{-8}{-8}{1}{-1}{2}.  \]

}


\frame{ \frametitle{Elementary row operations: scale one row}

To \textbf{scale row $k$ by the scalar $c$}, use the elementary matrix $E_{k,c}$, which is the modification of $I$ where the row $k$ diagonal is $c$ instead of 1. For example,  
\[ E_{2,8} = \mrrr{1}{0}{0}{0}{8}{0}{0}{0}{1} \text{ changes } A = \mrrr{3}{4}{5}{1}{-1}{2}{-6}{-8}{-8} \]
to 
\[ E_{2,8} A = \mrrr{1}{0}{0}{0}{8}{0}{0}{0}{1} \mrrr{3}{4}{5}{1}{-1}{2}{-6}{-8}{-8} = \mrrr{3}{4}{5}{8}{-8}{16}{-6}{-8}{-8}. \]

}


\frame{ \frametitle{Elementary row operations: scale row AND add to another}

To \textbf{add the $c$-scaled row $k$ to row $j$}, leaving only row $j$ changed, use the elementary matrix $E_{k,j,c}$, which is the modification of $I$ where the row $j$, column $k$ entry is changed from 0 to $c$. For example, $E_{1,2,-3}$ causes the change 
\[ \text{row } 2 \rightarrow -3 \cdot \text{row }1 + \text{row } 2: \]
\[ E_{1,2,-3} A = \mrrr{1}{0}{0}{-3}{1}{0}{0}{0}{1} \mrrr{3}{4}{5}{1}{-1}{2}{-6}{-8}{-8} = \mrrr{3}{4}{5}{-8}{-13}{-13}{-6}{-8}{-8}. \]

}


\frame{ \frametitle{Elementary matrices affect rows or columns}

An elementary matrix changes the \emph{rows} of a matrix if multiplying \emph{on the left}. 

\vspace{5mm}

An elementary matrix changes the \emph{columns} of a matrix if multiplying \emph{on the right}. 

\[ \text{row } 2 \rightarrow -3 \cdot \text{row }1 + \text{row } 2: \]
\[ E_{1,2,-3} A = \mrrr{1}{0}{0}{-3}{1}{0}{0}{0}{1} \mrrr{3}{4}{5}{1}{-1}{2}{-6}{-8}{-8} = \mrrr{3}{4}{5}{-8}{-13}{-13}{-6}{-8}{-8}. \]

\[ \text{col } 1 \rightarrow -3 \cdot \text{col }2 + \text{col } 1: \]
\[ A E_{1,2,-3} = \mrrr{3}{4}{5}{1}{-1}{2}{-6}{-8}{-8} \mrrr{1}{0}{0}{-3}{1}{0}{0}{0}{1} = \mrrr{-9}{4}{5}{4}{-1}{2}{18}{-8}{-8}. \]

}


\frame{ \frametitle{Elementary matrix inverses: swap back}

Each kind of elementary matrix has an inverse matrix that is also an elementary matrix of the same type.

\vspace{5mm}

To \textbf{swap two rows back}, use the \textbf{same elementary permutation matrix} - order of the row indices in the notation doesn't matter! 

\vspace{5mm} 

For example, to \textbf{swap two rows} 2 and 3 back to their original positions, just swap rows 3 and 2! Same action! 

\[ P_{j,k}^{-1} = P_{k,j} = P_{j,k} \]

\[ P_{2,3}^{-1} = P_{3,2} = P_{2,3} = \mrrr{1}{0}{0}{0}{0}{1}{0}{1}{0}. \]

}


\frame{ \frametitle{Elementary matrix inverses: scale one row}

To \textbf{undo scaling row $k$ by the scalar $c$}, use the elementary matrix $E_{k,\frac{1}{c}}$, which scales row $k$ by $\frac{1}{c}$. 
\[ E_{k,c}^{-1} = E_{k,\frac{1}{c}} \]
For example,  
\[ E_{2,8} = \mrrr{1}{0}{0}{0}{8}{0}{0}{0}{1}; \,\, E_{2,8}^{-1} = E_{2,\frac{1}{8}} = \mrrr{1}{0}{0}{0}{\frac{1}{8}}{0}{0}{0}{1}. \]

}


\frame{ \frametitle{Elementary matrix inverses: scale and subtract}

To \textbf{undo adding the $c$-scaled row $k$ to row $j$}, leaving only row $j$ changed, just subtract that scaled row: 
\[ E_{k,j,c}^{-1} = E_{k,j-c} \]
For example, 
\[ E_{1,2,-3} = \mrrr{1}{0}{0}{-3}{1}{0}{0}{0}{1}; \,\, E_{1,2,-3}^{-1} = E_{1,2,3} = \mrrr{1}{0}{0}{3}{1}{0}{0}{0}{1}. \]

}


\frame{ \frametitle{Solving a system of linear equations}

Now we will outline the process to solve an arbitrarily large square system of linear equations ($n$ equations in $n$ variables).

\vspace{5mm}

If this process works, not only will we solve the system of equations 
\[ A \vec{x} = b, \] 
but we will have a \textbf{factorization} of the matrix $A$ into the product of three matrices: 
\[ A = LDU, \]
where 
\begin{itemize}
\item $L$ is a \textbf{lower triangular} matrix with diagonal 1s and 0s above;
\item $D$ is a \textbf{diagonal} matrix, with nonzeros on the diagonal and 0s elsewhere;
\item $U$ is an \textbf{upper triangular} matrix with diagonal 1s and 0s below.
\end{itemize}

}


\frame{ \frametitle{Solving a system of linear equations}

First, we will assume this process works, and so there exists a factorization 
\[ A = LDU. \]
If this process fails at any point, then the rows of $A$ must be rearranged, or there is not a unique solution to the system.

\begin{itemize}
\item[1. ] Commit forward, downward elimination of variables by left multiplying elementary matrices on the equation 
\[ A \vec{x} = b, \,\,\text{ that is, } \,\, LDU\vec{x} = b, \] 
until it is in the form 
\[ DU \vec{x} = L^{-1} b. \]
\end{itemize}

}


\frame{ \frametitle{Solving a system of linear equations}

\begin{itemize}
\item[2. ] Scale each equation so that the leading variable on each is 1. 
This will change the system from 
\[ DU \vec{x} = L^{-1} c \]
into 
\[ U \vec{x} = D^{-1} L^{-1} b. \]
\item[3. ] Upward, backward substitution of variables, starting at the bottom, yielding the  solution 
\[ \vec{x} = U^{-1} D^{-1} L^{-1} b. \]
\end{itemize}

}


\frame{ \frametitle{Solving a system of linear equations, factoring $A = LDU$}

In other words, we compute the inverse matrix 
\[ A^{-1} = U^{-1} D^{-1} L^{-1} \]
and apply it to 
\[ A \vec{x} = b \]
to get 
\[ \vec{x} = A^{-1} b. \]
Since $L^{-1}$, $D^{-1}$, and $U^{-1}$ are all products of elementary matrices, we can find their inverses $L$, $D$, and $U$ to find the factorization 
\[ A = LDU. \]  

}


\frame{ \frametitle{Computing the solution, factorization of a linear system}

We will use the technique of an \textbf{augmented matrix} to do our calculations by hand.

\vspace{5mm}

This allows us to do computations on $A$ and $b$ simultaneously. 

\vspace{5mm}

We start with $A \vec{x} = b$ in the form 
\[ \left[ A \, \bigg| \, b \right] \]
and attempt to end with the solution $\vec{x} = A^{-1} b$ in the form 
\[ \left[ I \, \bigg| \, A^{-1} b \right] \]
by committing left multiplication of elementary matrices. 

}


\frame{ \frametitle{Computing using an augmented matrix}

We will call the diagonal entries on the left side \textbf{pivots}. 

\vspace{5mm}

Our goal is to make the pivots 1 (representing 1 of each variable) and 0 elsewhere on the left side, by eliminating variable coefficients below each pivot.

\vspace{5mm}

If ever a pivot is zeroed out, this process fails for the given row organization of $A$. 

}


\frame{ \frametitle{Computing using an augmented matrix}

If ever a 
\begin{center}
``0 = 0'' \\
(underdetermined system; infinitely many solutions)
\end{center} 
or 
\begin{center} 
``0 = 1'' \\
(overdetermined system; no solutions) 
\end{center} 
occurs, then \emph{no} reorganization of the rows of $A$ will work, and the system does not have a unique solution.

}


\frame{ \frametitle{Transposes}

If $v$ is a row vector, then its \textbf{transpose}, denoted $v^t$, is a column vector with the same entries. 

\vspace{5mm}

If $v$ is a column vector, then its \textbf{transpose} is a row vector. 

\vspace{5mm}

The dot product of two row vectors $v$ and $w$ is the same value as the matrix product $v w^t$. 

\vspace{5mm}

The dot product of two column vectors $v$ and $w$ is the same value as the matrix product $v^t w$. 

\vspace{5mm}

In general, the \textbf{transpose} of the $m \times n$ matrix $A$ is the $n \times m$ matrix $A^t$ that changes rows to columns.
}


\frame{ \frametitle{Properties of Transposes}

\begin{itemize}
\item $(A^t)^t = A$
\item $(A + B)^t = A^t + B^t$
\item $A$ is called a \textbf{symmetric matrix} if $A^t = A$. 
\item Diagonal matrices are clearly symmetric: $D^t = D$. 
\item If the product $AB$ exists, then $(AB)^t = B^t A^t$.
\item $A\vec{x} = b$ is the same system as $\vec{x}^t A^t = b^t$.
\end{itemize}

}


\frame{ \frametitle{Properties of Inverses}

\begin{itemize}
\item $(A^{-1})^{-1} = A$
\item $(A^{-1})^{t} = (A^t)^{-1}$
\item $A$ is called an \textbf{involution} if $A^{-1} = A$. This means $A^2 = I$. 
\item If the product $AB$ exists, and $A$ and $B$ are both invertible, then $(AB)^{-1} = B^{-1} A^{-1}$.
\item $A\vec{x} = b$ is solved by $\vec{x} = A^{-1} b$, if $A^{-1}$ exists.
\item If $A = LDU$ and $A$ is symmetric, then $U = L^t$, yielding the factorization $A = LDL^t$. 
\item Call an invertible square matrix $A$ \textbf{orthogonal} if $A^t = A^{-1}$. Then the columns $a_1, a_2, ..., a_n$ of $A$ are all orthonormal:
\[ A = \rvvvv{a_1}{a_2}{...}{a_n} \implies A^{-1} A = A^t A = I. \]
\end{itemize}

}


\frame{ \frametitle{Properties of Permutations}

\begin{itemize}
\item All permutation matrices $P$ are square. 
\item Permutation matrices are orthogonal.
\item If a square matrix $A$ is invertible, but does not allow our elimination process to work to factor $A = LDU$, then some permutation of the system $PA$ will. 

\vspace{3mm}

(We will not go through the process of seeing how to decide how to determine such a $P$ here.)
\end{itemize}

}


\end{document}
